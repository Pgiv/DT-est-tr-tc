\thispagestyle{fancy}

\hypertarget{ruxe9sumuxe9}{%
\section*{Résumé}\label{ruxe9sumuxe9}}
\addcontentsline{toc}{section}{Résumé}

Cette étude s'intéresse aux différentes méthodes de construction de
moyennes mobiles pour l'estimation en temps réel de la tendance-cycle et
la détection rapide des points de retournement. Elle montre comment
toutes les méthodes étudiées (filtres polynomiaux locaux, méthodes
basées sur une optimisation sous contrainte d'une somme pondérée de
critères de qualité des moyennes mobiles, et filtres basés sur les
espaces de Hilbert à noyau reproduisant (RKHS)) peuvent s'écrire comme
un cas particulier d'une formule générale de construction de moyennes
mobiles. Cela permet de mettre en avant les similitudes et les
différences entre les méthodes et ainsi les comparer plus facilement.
Cette étude montre également deux prolongements possibles aux filtres
polynomiaux locaux : l'ajout d'un critère permettant de contrôler le
déphasage (délai dans la détection des points de retournement) et une
façon de paramétriser localement ces filtres. Les méthodes sont
également comparées entre elles sur des séries simulées et réelles. Cela
permet de mettre en lumière l'impact de problèmes d'optimisation, dûs à
l'existence de plusieurs extremum, dans les filtres issus des RKHS sur
les révisions et le déphasage. Par ailleurs, cela montre que modéliser
des tendances polynomiales trop complexes introduit plus de révisions
sans diminuer le déphasage. Enfin, cela montre également comment, pour
les filtres polynomiaux locaux, une paramétrisation locale permet une
détection plus rapide des points de retournement et une réduction des
révisions des estimations de la tendance-cycle.

\hypertarget{abstract}{%
\section*{Abstract}\label{abstract}}
\addcontentsline{toc}{section}{Abstract}

This paper describes and compares different approaches to build
asymmetric filters: local polynomials filters, methods based on an
optimization of filters' properties (Fidelity-Smoothness-Timeliness,
FST, approach and a data-dependent filter) and filters based on
Reproducing Kernel Hilbert Space. It also describes how local
polynomials filters can be extended to include a timeliness criterion to
minimize phase shift. All these methods can be seen as a special case of
a general unifying framework to derive linear filters.

This paper shows that, when the length of the filter is adapted to the
variability of the series, constraining asymmetric filters to preserve
constant trends (and not necessarily polynomial ones) reduce revision
error and time lag. Therefore, future studies on the subject can focus
on these filters. Moreover, with RKHS filters some optimisation issues
can occurs and they might lead to erratic estimation. They might be able
to produce satisfying results in terms of phase-shift and revisions, but
they should be avoid for the last estimates of the trend-cycle
component: other methods should then be prefered to reduce revisions
with the final estimates.

All the methods are implemented in the \faIcon{r-project} package
\texttt{rjd3filters} and the results can be easily reproduced. The
programs used, and a web version of this report, are available at
\url{https://github.com/InseeFrLab/XXXXXXXXX}.

\newpage
