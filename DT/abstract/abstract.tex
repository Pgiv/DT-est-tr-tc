\thispagestyle{fancy}

\hypertarget{ruxe9sumuxe9}{%
\section*{Résumé}\label{ruxe9sumuxe9}}
\addcontentsline{toc}{section}{Résumé}

Cette étude s'intéresse aux différentes méthodes de construction de
moyennes mobiles pour l'estimation en temps réel de la tendance-cycle et
la détection rapide des points de retournement. Elle montre comment les
méthodes étudiées peuvent s'écrire comme un cas particulier d'une
formule générale de construction de moyennes mobiles, ce qui permet de
décrire leurs similitudes et ainsi les comparer plus facilement. Elle
décrit également deux prolongements possibles aux filtres polynomiaux
locaux : l'ajout d'un critère permettant de contrôler le déphasage
(délai dans la détection des points de retournement) et une façon de
paramétriser localement ces filtres. La comparaison des méthodes sur des
séries simulées et réelles montre que : les problèmes d'optimisation des
filtres issus des RKHS augmentent le déphasage et les révisions des
estimations de la tendance-cycle ; modéliser des tendances polynomiales
trop complexes introduit plus de révisions sans diminuer le déphasage ;
pour les filtres polynomiaux locaux, une paramétrisation locale permet
une réduction du déphasage et des révisions.

Mots clés : séries temporelles, tendance-cycle, désaisonnalisation
points de retournement.

\hypertarget{abstract}{%
\section*{Abstract}\label{abstract}}
\addcontentsline{toc}{section}{Abstract}

Cette étude s'intéresse aux différentes méthodes de construction de
moyennes mobiles pour l'estimation en temps réel de la tendance-cycle et
la détection rapide des points de retournement. Elle montre comment les
méthodes étudiées peuvent s'écrire comme un cas particulier d'une
formule générale de construction de moyennes mobiles, ce qui permet de
décrire leurs similitudes et ainsi les comparer plus facilement. Elle
décrit également deux prolongements possibles aux filtres polynomiaux
locaux : l'ajout d'un critère permettant de contrôler le déphasage
(délai dans la détection des points de retournement) et une façon de
paramétriser localement ces filtres. La comparaison des méthodes sur des
séries simulées et réelles montre que : les problèmes d'optimisation des
filtres issus des RKHS augmentent le déphasage et les révisions des
estimations de la tendance-cycle ; modéliser des tendances polynomiales
trop complexes introduit plus de révisions sans diminuer le déphasage ;
pour les filtres polynomiaux locaux, une paramétrisation locale permet
une réduction du déphasage et des révisions.

Mots clés : séries temporelles, tendance-cycle, désaisonnalisation
points de retournement.

JEL Classification: E32, E37.

\newpage
